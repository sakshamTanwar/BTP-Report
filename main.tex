\documentclass{article}
\usepackage[utf8]{inputenc}

\title{Format of BTP Report}
\author{Naman Jain, Pranshu Maheshwari, Saksham Tanwar}
\date{\today}

\begin{document}

\maketitle

\section*{Sections in Report}
\begin{enumerate}
    \item Introduction
    \begin{itemize}
        \item Description of a generic application
        \item Disadvantages of a centralized database layer
        \item Advantages of using a blockchain solution
        \item Introduction of our product, highlighting the importance of blockchain, describing overall architecture and various products used.
    \end{itemize}
    \item Introduction of essential Concepts of Blockchain
    \begin{itemize}
        \item Describe various offerings of blockchain in regards to our product
        \item Existing applications related to our product
    \end{itemize}
    \item Architecture
    \begin{itemize}
        \item Details regarding the solution architecture
        \item Identifying the various actors/users of the product and how they interact with the product
        \item Various sub modules of the product and how they interact with one another
    \end{itemize}
    \item Deployment Details
    \begin{itemize}
        \item Details about how the product can be deployed
    \end{itemize}
    \item Conclusion
    \begin{itemize}
        \item Identify the problems which the product solved
        \item Identify the problems still left in the solution proposed
    \end{itemize}
    \item References

\end{enumerate}

\section{Introduction}
    \subsection{Generic Application}
        \paragraph{} A typical application can be broken down into three logical layers namely:-
        \begin{itemize}
            \item \textbf{Presentation Layer: }It is the layer responsible for display information and collect information from the user.
            \item \textbf{Business Logic Layer: }In this layer, information collected from presentation layer is used to perform the various calculations and operations needed to be performed by the application.
            \item \textbf{Data Layer: }This layer is responsible for storing and retrieving data to be used by other layers.
        \end{itemize}

        \paragraph{}Data Layer is typically implemented using a database, which is an organized collection of data. A database can be of two types : Centralized and Distributed.
        
    \subsection{Disadvantages of a centralized database}
        \paragraph{}
        Centralized database is a database that is located, stored, and maintained in a single location.
        A distributed database is a database in which all the information is stored on multiple physical locations. There are two types of distributed databases homogeneous and heterogeneous. In a homogeneous distributed database all the locations store the database identically i.e. all locations have the same management system and schema. Whereas in heterogeneous distributed database different locations can have different management software, schemas.
        A decentralized database is a database which doesn't have a central owner, it uses multiple central owners where each central owner has a copy of the data which can be accessed by the users. 
        \cite{wiki}
        \paragraph{}
        If a centralized database is used to store the data then the whole system is prone to Single Point of Failure, if for some reason the database fails the whole system will fail. As all the data is stored in a single location, if there are precautionary measures taken then a hardware failure can lead to complete data loss. The databases are also vulnerable to cyberattacks which can cause data loss, data leaks, data inconsistency and many other vulnerability.
        
    \subsection{Advantages of using Blockchain}
        \paragraph{}
        % general introduction about blockchain benefits.
        Blockchain is a type of decentralized database where the data is stored in blocks and each block is linked to the previous block using its cryptographic hash, thus forming a chain. As the blockchain grows in size, it becomes more and more difficult to tamper the data in these blocks. Blockchain based databases are more trustworthy, reliable and secure than traditional databases.
        
        \paragraph{}
        % decentralization benefits
        A blockchain based system is decentralized. Hence, a centralized authority is not necessary. Decentralized systems are also resilient to single point of failure. A blockchain network comprises of various nodes, each maintaining its own copy of database. Hence, all the data is replicated, shared and synchronized across all the nodes in the network. This makes the data almost tamper proof as it will be very costly and thus highly unlikely for a malicious person to change the data across the whole network. No node can directly write data to the blockchain as it can make the data inconsistent, to avoid this the blockchain network uses a consensus algorithm which helps all the nodes in the network to come to a consensus and commit the data in the blockchain. This helps to keep all the nodes in the network to get synchronized with each other. 

        \paragraph{}
        % Immutability and tracability benefits.
        The data that is committed to the blockchain is immutable. After some data is inserted into the blockchain, it is very difficult to delete or alter it. This is realized by including, in each block, the cryptographic hash of its previous block. Because of this immutability, a blockchain database only supports read and write operations in contrast to traditional databases which supports read, write, update and delete operations. To update the database state, a subsequent write is required. Old writes persist in the blockchain forever, making it convenient to trace how the blockchain database state is updated over time. This makes it possible to trace the actions that resulted in a particular database state.

        \paragraph{}
        % transparency
        Another benefit of blockchain is that it provides transparency to the data stored in it. This is achieved by replicating the blockchain data among multiple nodes present in the blockchain network. Nodes in the blockchain network not only can view the data already committed to the blockchain but can also participate in the process of validating the data to be committed in the blockchain. The transparency also helps users to cross verify some data against the blockchain.  

        \paragraph{}
        % single public ledger
        A blockchain ledger is logically a single public ledger. Since every node updates its copy of blockchain ledger in sync with each other and after consensus is reached, the whole network logically act as one single public ledger. This removes the complications present in traditional systems where different participants or organizations maintains multiple ledgers which needs to be reconcile and synchronized time to time.
        
        \paragraph{}
        % Reduced Cost
        Systems that are deployed on public blockchain network are more cost efficient as compared to the traditional systems. This is because the system utilizes the processing power and resources of a large number of nodes that are already connected to the blockchain network, thus significantly reducing the cost needed for setting up and maintaining centralized servers present in traditional systems.

        \paragraph{}
        % Security
        Blockchain based systems are more secure and resilient to cyber attacks, which can cause data loss, data leaks, data inconsistency and many other vulnerability, than traditional sytems. Whenever a new block is introduced in the blockchain, its hash value is calculated which also includes the hash of its previous block. If a malicious user fraudulently tries to tamper the data inside a block, not only its hash value changes, but the hashes of the following blocks get changed too. This, along with the fact, that the data is stored in multiple nodes present in the blockchain network, makes it very difficult to tamper the data present in the blockchain.


        \paragraph{}
        % Benefits of blockchain in our solution
        
    \subsection{Motivation}
        \paragraph{}
        In India, currently if person is asked to prove if he/she is the owner of a piece of land then all they can show for it is a sale deed which just proves that the person was the owner at a particular time, but that person cannot prove that he/she is still the owner of that land. For proving the same he/she has to go to various government offices and collect various document showing that no sale deed has been registered for that land after the one which the person has. When a person wishes to buy land in India, they have to be very careful and perform various checks such as
        \begin{itemize}
            \item Check if the deed title is in the name of the seller and if he/she has the full right to sell it
            \item Procure a Encumbrance Certificate from sub-registrar's office where the deed is registered which declares that the land is free of any legal hassle and unpaid dues
            \item Check if the Property tax and other bills are paid in full and the seller has the respective original receipts
            \item Check if the loan on the land has been completely repaid
        \end{itemize}
        If a buyer is lethargic in verifying these documents he/she can be easily duped into buying a disputed land parcel or it may even happen that the land in question did not even legally belong to the said seller, that is the seller was not the genuine owner of that land provided fake documents. As these documents are mostly maintained offline as hard copies these cases occur often.

        \paragraph{}
        Getting all the information such as the sale deeds and ownership history for a land parcel is very cumbersome, a simple solution to this problem is a smartphone application with which users/potential buyers can get land ownership history and related documents for a particular land parcel. These documents and data can be stored in a database by the government and will be updated every time a new sale deed is registered i.e. owner for a land parcel changes. This makes the process of acquiring the documents in question very easy and hassle free.
\end{document}
