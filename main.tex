\documentclass{article}
\usepackage[utf8]{inputenc}
\usepackage[english]{babel}

\usepackage[nottoc]{tocbibind}

\title{BTP Report}
\author{Naman Jain, Pranshu Maheshwari, Saksham Tanwar}
\date{\today}

\begin{document}

\maketitle
\clearpage
\tableofcontents
\clearpage

% \section*{Sections in Report}
% \begin{enumerate}
%     \item Introduction
%     \begin{itemize}
%         \item Description of a generic application
%         \item Disadvantages of a centralized database layer
%         \item Advantages of using a blockchain solution
%         \item Introduction of our product, highlighting the importance of blockchain, describing overall architecture and various products used.
%     \end{itemize}
%     \item Introduction of essential Concepts of Blockchain
%     \begin{itemize}
%         \item Describe various offerings of blockchain in regards to our product
%         \item Existing applications related to our product
%     \end{itemize}
%     \item Architecture
%     \begin{itemize}
%         \item Details regarding the solution architecture
%         \item Identifying the various actors/users of the product and how they interact with the product
%         \item Various sub modules of the product and how they interact with one another
%     \end{itemize}
%     \item Deployment Details
%     \begin{itemize}
%         \item Details about how the product can be deployed
%     \end{itemize}
%     \item Conclusion
%     \begin{itemize}
%         \item Identify the problems which the product solved
%         \item Identify the problems still left in the solution proposed
%     \end{itemize}
%     \item References

% \end{enumerate}

\section{Introduction}
    \subsection{Generic Application}
        \paragraph{} A typical application can be broken down into three logical layers namely:-
        \begin{itemize}
            \item \textbf{Presentation Layer: }It is the layer responsible for display information and collect information from the user.
            \item \textbf{Business Logic Layer: }In this layer, information collected from presentation layer is used to perform the various calculations and operations needed to be performed by the application.
            \item \textbf{Data Layer: }This layer is responsible for storing and retrieving data to be used by other layers.
        \end{itemize}

        \paragraph{}Data Layer is typically implemented using a database, which is an organized collection of data. A database can be of two types : Centralized and Distributed.
        
    \subsection{Disadvantages of a centralized database}
        \paragraph{}
        Most applications require data in some form or another and need to store the said data for future use, this is where databases come into picture.
        A database is an organized collection of data, generally stored and accessed electronically from a computer system.
        Access to this data is usually provided by a "database management system" (DBMS) consisting of an integrated set of computer software that allows users to interact with one databases and provides access to all of the data contained in the database.
        DBMS's provide various functions that allow management of a database and its data such as Data definition which defines the database model i.e. the logical structure of the data model, Updating of actual data which includes insertion, modification, and deletion of the actual data, Retrieval of actual data which may or may not include data processing and Administration which includes allowing access to various users, enforcing data security and various other checks.
        \cite{wiki}
        A database model is a type of data model that determines the logical structure of a database and fundamentally determines in which manner data can be stored, organized and manipulated. The most popular example of a database model is the relational model, which uses a table-based format. The ACID (Atomicity, Consistency, Isolation and Durability) database design is one of oldest and most important database design. A relational database that fails to follow the ACID model cannot be considered reliable. 
        \paragraph{}
        Databases can categorized in many categories depending on how they store data, where it is stored and the additional functionality the DBMS can provide on the stored data. Broadly databases can categorized into two major categories centralized and distributed. They both have their advantages and disadvantages.
        A centralized database is a database that is located, stored, and maintained in a single location and has a single DBMS but can be accessed by the users distributed in the network.
        A distributed database system consists of a collection of local databases,  geographically located in different points (nodes of a network of computers) and  logically related by functional relations so that they can be viewed globally as a single database \cite{distributeddatabase}.
        There are two types of distributed databases homogeneous and heterogeneous. In a homogeneous distributed database all the locations store the database identically i.e. all locations have the same management system and schema. Whereas in heterogeneous distributed database different locations can have different management software, schemas.
        For a database management system to be distributed, it should be fully compliant with the twelve rules introduced by C.J. Date in 1987 \cite{distributeddbms}: local autonomy; the absence of a dependency from a central location; continuous operation; location independent; fragmentation independent; replication  independent; distributed query processing; distributed transaction management; hardware independent; operating system independent; independent of communication  infrastructure; independent of database management system.
        
        \paragraph{}
        If a centralized database is used to store the data then the whole system is prone to Single Point of Failure, if for some reason the database fails the whole system will fail. As all the data is stored in a single location, if there are no  precautionary measures taken then a hardware failure can lead to complete data loss. The databases are also vulnerable to cyberattacks which can cause data loss, data leaks, data inconsistency and many other vulnerability. This makes centralized database dubious with very low reliability. But at the same time, centralized database ensures data consistency and easy management in compliance with ACID design \cite{centralizeddistributeddatabases}.
        
        \paragraph{}
        A major advantage of distributed database is that by sharing a database across  multiple nodes can obtain a storage space extension and also can benefit from multiple processing resources. A distributed database system is robust to failure to some extent. Hence, it is reliable when compared to a  centralized database system. It is also more robust compared to a centralized database as it doesn't have a single point of failure, if a node fails another node or group of nodes can provide the necessary data. But to make this possible complex software's are required which incur additional costs and processing overheads. Marinating data integrity is difficult and hence minimum redundancy and ACID properties are more relaxed than compared to a centralized database \cite{centralizeddistributeddatabases}. Distributed environment also faces problems such as fragmentation and data replication. A data fragment constitutes some subset of the original database. A data replica constitutes some copy of the whole or part of the original database. The fragmentation and the replication can be combined: a relationship can be partitioned into several pieces and can have multiple replicas of each fragment \cite{distributedsystems}.
        
    \subsection{Advantages of using Blockchain}
        \paragraph{}
        Blockchain is type of database where data is stored in blocks and then the blocks are chained together. When a block fills up, a new block is created and filled with data and it is also chained to the previous block which makes the chained together. Data is blockchain in immutable i.e. blockchain only supports read and write operation rather than the tradition read, write, update and delete this makes the data not susceptible to change. Blockchain uses distributed ledger design i.e. all the data is replicated, shared and synchronized across all the nodes in the network. This makes the data almost tamper proof as it will be very costly and thus highly unlikely for a malicious person to change the data across the whole network. Blockchain also offers transparency as any node can read, and verify the data. No node can directly write data to the blockchain as it make the data inconsistent, to avoid this blockchain uses an consensus algorithm where, when data is to be added to the blockchain all the node come to a consensus according to the algorithm used and only then the data is added and synchronized in the chain.
        \paragraph{}
        Although blockchain has redundant data and is slower compared to the traditional databases, it provides the users with a secure, reliable, distributed, decentralized, immutable storage of data. On basis of application when trust is more important than speed blockchain provide an effective solution.
        
    \subsection{Motivation}
        \paragraph{}
        In India, currently if person is asked to prove if he/she is the owner of a piece of land then all they can show for it is a sale deed which just proves that the person was the owner at a particular time, but that person cannot prove that he/she is still the owner of that land. For proving the same he/she has to go to various government offices and collect various document showing that no sale deed has been registered for that land after the one which the person has. When a person wishes to buy land in India, they have to be very careful and perform various checks such as
        \begin{itemize}
            \item Check if the deed title is in the name of the seller and if he/she has the full right to sell it
            \item Procure a Encumbrance Certificate from sub-registrar's office where the deed is registered which declares that the land is free of any legal hassle and unpaid dues
            \item Check if the Property tax and other bills are paid in full and the seller has the respective original receipts
            \item Check if the loan on the land has been completely repaid
        \end{itemize}
        If a buyer is lethargic in verifying these documents he/she can be easily duped into buying a disputed land parcel or it may even happen that the land in question did not even legally belong to the said seller, that is the seller was not the genuine owner of that land provided fake documents. As these documents are mostly maintained offline as hard copies these cases occur often.

        \paragraph{}
        Getting all the information such as the sale deeds and ownership history for a land parcel is very cumbersome, a simple solution to this problem is a smartphone application with which users/potential buyers can get land ownership history and related documents for a particular land parcel. These documents and data can be stored in a database by the government and will be updated every time a new sale deed is registered i.e. owner for a land parcel changes. This makes the process of acquiring the documents in question very easy and hassle free.
    
    \clearpage
    \bibliographystyle{unsrt}
    \bibliography{citation}
\end{document}
